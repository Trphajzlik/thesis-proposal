\chapter{State of the Art}\label{chap:state-of-the-art}

A modular robotics is a way to build robots consisting of \emph{modules}. In
context of this paper, modules are rather high-level pieces with a certain level
of self-control instead of low-level components like individual actuators or
sensors. It might even make sense to talk about modules as individual robots,
which are used to build bigger robots. The modules have usually a limited set of
capabilities -- often the functionality is so primitive, the modules cannot
perform useful tasks on their own. However, when we join multiple modules, they
can cooperate and, therefore, new capabilities of the system as a whole emerge.
This idea probably first appeared in the work by
\textcite{DBLP:conf/icra/FukudaK90}, where they introduced CEBOTs (Cellular
Robotic System).

\section{Taxonomy of Existing Modular Robots}

Since the era of CEBOTs, many more projects followed and the research area split
into two: smart (or programmable) matter and modular robotics. The smart matter
aims at building very simple modules leveraging physical and chemical principles
(imagine artificial atoms) in order to create blob of modules, which can
reassemble into a different, usually solid, object based on an external input.
Contrary, the modular robotics aims at building more complex modules, which are
able of sophisticated self-organization (imagine artificial cells), which can
form usually highly dynamic robots that can autonomously move and interact with
their environment. Smart matter also aims at sub-millimeter modules, modular
robotics aims at sub-centimeter modules \cite{DBLP:conf/ieeealife/Christensen07,
1285597}. In the rest of this paper, we will omit smart matter and focus
exclusively on modular robots.

What distinguishes modular robots from swarm robotics is the ability of the
modules to mechanically connect and inherently form a larger robot. The
connection can be performed externally, e.g., by an operator, or the modules can
connect on their own. Even the trend is to build self-reconfigurable robots,
there are occasionally some exceptions, e.g., PetRo
\cite{DBLP:conf/ro-man/Salem14}.

There have been many more or less successful projects of self-reconfigurable
modular robots since the CEBOTs. The projects feature various designs approaches
from the capabilities of individual modules to the topology in which the modules
connect. Most of the projects used metamorphic modules, i.e., there is only a
single or a few types of modules in the system. We consider the following list
as a nice representable sample of the various designs: Fracta
\cite{DBLP:conf/icra/MurataKK94}, Molecule \cite{DBLP:conf/icra/KotayRVM98},
Polybot \cite{DBLP:conf/icra/YimDR00}, M-TRAN
\cite{DBLP:conf/icarcv/KurokawaKYTMK02}, Atron
\cite{DBLP:conf/iros/JorgensenOL04}, Superbot \cite{DBLP:conf/iros/SalemiMS06},
Molecube \cite{DBLP:journals/trob/ZykovMDL07}, Roombots
\cite{DBLP:conf/icra/SprowitzBDI09}, Symbricator
\cite{DBLP:journals/corr/abs-1109-2288}, SMORES \cite{DBLP:conf/iros/DaveyKY12},
M-Blocks \cite{DBLP:conf/iros/RomanishinGR13}, ModRED
\cite{DBLP:journals/ras/BacaHDND14}, HyMod \cite{DBLP:conf/dars/ParrottDG16} and
Omni-Pi-tent \cite{DBLP:conf/taros/PeckTT19}.

\textcite{4141032} distinguishes three categories of the robots based on the
topology in which they connect. Each category has its own specific problems of
control and makes some tasks easier than other:

\paragraph{Lattice architecture} have its modules arranged in regular 3D grid --
e.g., a cube or hexagonal grid. The modules are tightly packed together. Typical
examples of such architectures are M-Blocks \cite{DBLP:conf/iros/RomanishinGR13}
and Atron \cite{DBLP:conf/iros/JorgensenOL04}. The regular grid makes it easier
to create a reconfiguration schedule and execute in parallel (for more details
see Subsection \todo{ref}). However, reconfiguration is usually the only option
for locomotion of such system, and, therefore, lattice architectures do not
yield highly dynamic systems.

\paragraph{Chain architecture} have its modules connected in a string or
possibly in a tree. Typical examples of such architectures are Polybot
\cite{DBLP:conf/icra/YimDR00} and Molecubes
\cite{DBLP:journals/trob/ZykovMDL07}. This architecture allows for easy
formation of limbs and arms, therefore, these robots usually interact well with
the environment. Also even very simple controllers yield locomotion (via
snake-like movements (see Subsection \todo{ref} for more details). The chains
can also form space-filling curve, therefore, e.g., Molecubes, can form a
lattice-like structures, while the underlying structure is linear.

\paragraph{Hybrid architecture} allows for both arrangements of modules,
therefore combines the advantages of both, possibly at the cost of increased
complexity. Examples of such robots are M-TRAN~III
\cite{DBLP:journals/ijrr/KurokawaTKKHM08}, Roombots
\cite{DBLP:conf/icra/SprowitzBDI09}, SMORES \cite{DBLP:conf/iros/DaveyKY12},
HyMod \cite{DBLP:conf/dars/ParrottDG16} and Omni-Pi-tent
\cite{DBLP:conf/taros/PeckTT19}. We can perceive that the most recent trend is
to build hybrid architectures. Especially SMORES are designed to be able to
replicate arrangements of the other platforms, thus be as versatile as possible
\cite{DBLP:conf/iros/DaveyKY12}.

On top of the locomotion provided by the the inter-module interaction, some
systems feature locomotion of the individual modules - e.g, by providing wheels
(SMORES, HyMod) or tracks (Symbricator), which further makes the reconfiguration
problem easier.

\section{Planning and Control Challenge}

\todo{There are some common challenges in robotics in general and modular robotics}

\todo{Let's focus on the special ones for modular robots}

\todo{Illustrate it on the problem of finding a dock to mate}

\todo{Explain, why central control is bad}

\todo{Show, how the robots can communicate and why sometimes it is bad.}

\todo{Explain difference between reconfiguration and locomotion, say you will show examples later}

\todo{Explain, common approches to controlling the robots in a distributed
    manner. Present case studies and experiments in the following points:}

\todo{Present gait tables and basic synchronization approaches for them }

\todo{Present digital hormones}

\todo{Present nested controllers}

\todo{Also present the central solutions tackling the grand challenges - e.g. the SMORES navigation paper }


\section{Fault-tolerance Challenge}

There is no single widely accepted interpretation of what fault tolerant systems
should comply to. Usually, in a context of embedded system, a system is
considered fault-tolerant when it is able to continue operating (possibly with
worse performance) after one of its components fails
\cite{DBLP:journals/micro/Johnson84}. The extent of fault-tolerance depends on
what type of failures the system survives.

For example, we usually consider living organisms as highly fault-tolerant
systems. If an animal looses an limb, it is able to adapt and perform similar
tasks. E.g., a cat without a leg is still able to move and climb -- possibly
slower, but it is able to survive.

In the context of metamorphic robots, \textcite{DKbotDistr} discuss several
aspects of fault-tolerance and for some of them they propose solutions. They
distinguish several types of fault:

\paragraph{Complete module malfunction,} where the module stops responding and
acts passively. The usual assumption is that the other modules in the system can
detect such module. This type of malfunction is probably the most mentioned one
in the literature. The usual solution is to either drop or to ignore such
modules and reconfigure into a new configuration
\cite{DBLP:conf/ieeealife/Christensen07, DMotionCoord}.

\paragraph{Byzantine module malfunction,} where it is not easy to detect the
module has failed -- e.g., the module can have a wrong sensor and it can report
wrong data to other modules. The module can even be malicious. There is not much
work on this type of malfunction in the context of metamorphic robots as far as
we are aware. However, in context of distributed systems in general, it is a
vivid research topic. E.g., work by \textcite{DBLP:conf/osdi/CastroL99} could be
adapted for metamorphic robots.

\paragraph{Actuator malfunction,} where the module control unit is fully
functional, however, one or more of the modules actuators are either stuck
in a position or spin freely. This type of malfunction was tackled by
\textcite{DBLP:conf/romoco/VonasekONW15}.

\paragraph{Explosion of the whole system} is a type of malfunction where the
robot is broken down into pieces randomly spattered over the environment usually
after a high energy impact. \textcite{DBLP:conf/iros/YimSSPDT07a} proposed a
solution for automatic reassembly after explosion. They also discuss, that
system explosion is a way of fault-avoidance. When a system can reassemble, it
is desirable to include weak, re-attachable joints, which protect the individual
modules.

\todo{This section misses more in-depth references for existing solutions}

\section{Challenge of Reconfiguration}

\todo{Present the basic types of tackling the problem}
\todo{ Present state-space exploration and its optimization }
\todo{ Show the old papers and results about meta-modules }
\todo{ Show the "ultimate" paper claiming M-TRAN reconfiguration is NP hard }
\todo{ Somewhere show the relation to inverse kinematics and the new SMORES paper}

\section{Challenge of Locomotion}

\todo{Not sure if this will be covered by control and planning or not}
\todo{Present the special kinematics}

\section{Mixed Software and Hardware Challenges}

\todo{Raise question about hardware reliability}
\todo{Raise question about firmware distribution}
\todo{Raise question about security of such robots}

% The system can be \emph{centrally controlled} by a single (and possibly
% external) unit, or the distributed nature of the modules can be leveraged, and
% therefore, the system can feature \emph{distributed control}. The centrally
% controlled approach is considered as an easier one; however, it does not utilize
% all the potential computational power of the modules and it is harder to make
% the system fault-tolerant (due to the presence of a single point of failure in
% the form of the control unit) compared to the distributed control.