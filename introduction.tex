\chapter{Introduction}\label{chap:introduciton}

Metamorphic self-reconfigurable robotics is one of the future challenges the
field of robotics is facing. This field aims to design and develop an
autonomous robotic construction kit, which would allow for the construction of
sophisticated robots designed for specific tasks from smaller, simpler, and
unified autonomous units (modules). The advantage of such an approach lies in
its universality -- a single collection of modules can be reused multiple times
to fulfill a plethora of tasks.

Moreover, the specific feature of metamorphic robots is that many of connected
modules may autonomously change their mutual position and interconnection to
take a different physical shape, the ability referred to as
self-reconfiguration. As a result, various tasks may be completed within
a~single mission by a single robot as it appropriately adapts its shape to fit
the individual mission requirements. Also, the self-reconfiguration property
could bring us closer to the rapid prototyping of robots. We could treat robots
just like we treat software nowadays -- the developers just write robot
description, and by \say{compiling} it, the robot self-assembles. Such ability
would also allow for version-control of robots with an easy option to roll-back
to an older revision without the necessity of keeping a physical inventory of
older versions.

Metamorphic robots might also be a solution for building fault-tolerant robots.
A robotic system built out of a large number of uniform modules could easily
recover from a module failure, just like a living organism survives the death of
individual cells.

Although the idea of metamorphic robots seems to be an appealing approach to
solve a variety of problems, the current state-of-the-art solutions are far from
practical applications. There are still too many unresolved problems ranging
from theoretical to practical ones. These include, for example, the absence of
an efficient search for a reconfiguration plan, the lack of a unified approach
to modules' mass control, or the process of space miniaturization and cost
reduction \cite{4141032}.

\textcite{4141032} formulated challenges of the field of that time -- this
includes hardware design, planning, and control challenges. Most of the
challenges withstand to date. A few years later,
\textcite{DBLP:journals/corr/abs-1108-5543} proposed the \emph{first grand
challenge of modular robots}; after which completion, we could consider advances
in this field sufficient for practical applications. They proposed a scenario
where 100 modules are placed in an unknown area with walls and obstacles for 100
days. The modules are required to remain active during the whole time without
any human intervention. The environment provides power sources, however, they
are not directly reachable by a~single module. Also, the location and the number
of power sources slowly change in time, so the robotic system has to adapt. To
our knowledge, nobody was able to finish the challenge so far. Nevertheless, the
research in this field is vivid, and new results come every year.

While we can find results that deal with the theory of metamorphic robots and
works that deal with the physical implementation of metamorphic modules or
platforms, we see the advancement in this field quite splattered and incoherent.
Research results that would interconnect both aspects of the problem appear
rarely. The situation is caused mainly by the fact that the physical design of
metamorphic modules is, in many cases, publicly unavailable, hence
irreproducible. As a result, researchers outside the original team of authors
cannot reuse the platform and build their research on top of the previously
known results.  Any reproducible and trustworthy research thus has to be made
either from scratch or based only on some guessed assumptions of what is
practical and realistic in case the researcher has no physical platform at hand.
Both cases significantly damper the research in the field.

In the rest of the thesis proposal, we first provide a general overview of the
area of metamorphic self-reconfigurable robots and present state-of-the-art in
the relevant subareas in Chapter \ref{chap:state-of-the-art}. Then we present
the selected challenges in the area of distributed and fault-tolerant control of
metamorphic robots we would like to tackle during our PhD study. We also give a
time schedule for the study in this chapter. In the Chapter \ref{chap:results}
we describe results we achieved so far and present \emph{RoFI} -- a new platform
of distributed metamorphic robots that is open-hardware and open-source, hence
almost freely available to anyone interested. The motivation behind this
platform is mainly to bridge the gap between theoretical and practical research
in the area.
